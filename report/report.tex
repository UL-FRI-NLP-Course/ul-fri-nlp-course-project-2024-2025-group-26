%%%%%%%%%%%%%%%%%%%%%%%%%%%%%%%%%%%%%%%%%
% FRI Data Science_report LaTeX Template
% Version 1.0 (28/1/2020)
% 
% Jure Demšar (jure.demsar@fri.uni-lj.si)
%
% Based on MicromouseSymp article template by:
% Mathias Legrand (legrand.mathias@gmail.com) 
% With extensive modifications by:
% Antonio Valente (antonio.luis.valente@gmail.com)
%
% License:
% CC BY-NC-SA 3.0 (http://creativecommons.org/licenses/by-nc-sa/3.0/)
%
%%%%%%%%%%%%%%%%%%%%%%%%%%%%%%%%%%%%%%%%%


%----------------------------------------------------------------------------------------
%	PACKAGES AND OTHER DOCUMENT CONFIGURATIONS
%----------------------------------------------------------------------------------------
\documentclass[fleqn,moreauthors,10pt]{ds_report}
\usepackage[english]{babel}

\graphicspath{{fig/}}




%----------------------------------------------------------------------------------------
%	ARTICLE INFORMATION
%----------------------------------------------------------------------------------------

% Header
\JournalInfo{FRI Natural language processing course 2025}

% Interim or final report
\Archive{Project report} 
%\Archive{Final report} 

% Article title
\PaperTitle{Automatic generation of Slovenian traffic news for RTV Slovenija} 

% Authors (student competitors) and their info
\Authors{Janez Kuhar, Veljko Dudić, and Marko Rozman}

% Advisors
\affiliation{\textit{Advisors: Slavko Žitnik}}

% Keywords
\Keywords{}
\newcommand{\keywordname}{Keywords}


%----------------------------------------------------------------------------------------
%	ABSTRACT
%----------------------------------------------------------------------------------------

\Abstract{
In this report we present the related works, initial ideas and proposed dataset for our project.
}

%----------------------------------------------------------------------------------------

\begin{document}

% Makes all text pages the same height
\flushbottom 

% Print the title and abstract box
\maketitle 

% Removes page numbering from the first page
\thispagestyle{empty} 

%----------------------------------------------------------------------------------------
%	ARTICLE CONTENTS
%----------------------------------------------------------------------------------------

\section*{Introduction}
The project focuses on replacing the manual production of traffic reports with an automated system. Traffic data provided in tabular form and traffic data from the promet.si website is used to generate traffic reports according to provided specifications.

\subsection*{Motivation}

\textbf{Timely Information.} Reliable traffic updates are critical for public safety and efficient transportation management.\\
\textbf{Manual Process Limitations.} Currently, students manually verify and type these reports every 30 minutes, leading to potential delays and inaccuracies.\\
\textbf{Automated Solution.} By utilizing LLMs and advanced prompt engineering (inspired by works on news headline generation), we aim to generate clear, concise, and contextually accurate traffic news.

\section{Related Work}

Li et al. \cite{liHGNewsNewsHeadline2021} proposed a decoder-only model for headline generation,
incorporating multihead attention, sentiment features, and part-of-speech information. Its focus
on generating succinct news headlines while accurately handling out-of-vocabulary
terms (such as specific road names) can be directly applicable to our task.

In a more recent work on the topic of text summarization \cite{houAlgorithmAutomaticAbstract2024}, a ChatGPT-based
algorithm for summarizing Russian texts was proposed. The suggested approach combined preprocessing (e.g., word segmentation,
stop word removal) with BERT for contextual understanding and a pointer mechanism for out-of-vocabulary
words.

Sha et al. \cite{sha2018order} proposed an order-planning mechanism for text generation from tabular data, combining
link-based attention with traditional content-based attention to model the sequence of information. Their approach
was further enhanced by a copy mechanism to handle rare words, improving the model's ability to generate accurate
and coherent summaries from structured data.

\section{Initial ideas}
\begin{itemize}
    \item Initial Prompt Engineering: Experiments will start by generating news texts directly from the structured traffic data using prompts.
    
    \item Selection of important data. Our observations show that the final news report is mostly made from only few important data points from the Excel file. We intend to filter the data to remove duplicate enteries and automaticly select the important data points relavant for our traffic news generation.
    
    \item Enhanced Generation through Fine-Tuning: The project will then incorporate parameter-efficient fine-tuning (e.g., LoRA) and retrieval techniques to further refine the generated content, ensuring correct road naming and event descriptions.
    
    \item Evaluation: A robust evaluation framework will be established using both automatic metrics (such as ROUGE, precision, recall, and F1) and human judgment to verify that the generated texts meet RTV Slovenija's standards.

    \item Pointer Mechanisms: Integrate pointer mechanisms to ensure that domain-specific terminology (e.g., road names and traffic event descriptors) is accurately reproduced, minimizing the risk of omitting critical details.

    \item Advanced Preprocessing: Apply rigorous preprocessing techniques—such as tokenization, normalization, and filtering—to both the traffic data and guideline documents. This will help standardize the input and ensure adherence to established formats.

    \item N-gram Language Features: Incorporate n-gram features to improve language fluency and ensure that generated sentences are coherent and stylistically consistent with existing RTV Slovenija news.

    \end{itemize}

\section{Methods}

\subsection*{Project dataset}

The primary dataset was provided to us by the TA. It consists of three important sets of files:

\begin{enumerate}
    \item \textbf{A structured Excel }file containing historical traffic report information from \textit{promet.si}.
    \item \textbf{Word documents} with lexical notes and the prescribed structure of a traffic report.
    \item \textbf{Ground truth data} in Rich Text Format (RTF) files, providing text in a ready-to-use format for radio hosts.
\end{enumerate}

% ToDo - verjetno nov subsetction "preprocessing"
% We intend to normalize the data by enforcing standard naming conventions, filtering out irrelevant details,
% and tagging urgent events (e.g., \textit{nujna prometna informacija}).


%----------------------------------------------------------------------------------------
%	REFERENCE LIST
%----------------------------------------------------------------------------------------
\bibliographystyle{unsrt}
\bibliography{report}


\end{document}
